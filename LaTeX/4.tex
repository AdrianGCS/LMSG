\documentclass[10pt,a4paper]{article}
\usepackage[utf8]{inputenc}
\usepackage{amsmath}
\usepackage{amsfonts}
\usepackage{amssymb}
\title{4}
\usepackage{color}
\begin{document}
	\section{colores con nombre}
		Este texto se dibuja \textcolor{red}{en rojo}
		\begin{itemize}
			\item \textcolor {green}{Verde}
			\item \textcolor {blue}{azul}
			\item \textcolor {red}{rojo}
			\item \textcolor {magenta}{Magenta}
		\end{itemize}
	\section{Colores sin\bigskip  nombre con RGB}
		
		\begin{itemize}
			\item \textcolor[rgb]{0,1,0}{Verde}
			\item \textcolor[rgb]{1,1,0}{Amarillo}
			\item \textcolor[rgb]{0.7 ,0.9, 0.3}{Amarillo}
			\item \textcolor[rgb]{1,1,1}{Blanco}
			\item \textcolor[rgb]{0,0,1}{azul}
			\item \textcolor[rgb]{1,0,1}{Magenta}
			\item \textcolor[rgb]{1,0,0}{Rojo}
			\item \textcolor[rgb]{0,1,1}{Cian}
		\end{itemize}
	\section{Colores sin nombre con CMYK}
		\begin{itemize}
			\item \textcolor[cmyk]{0,1,1,0}{Rojo}
			\item \textcolor[cmyk]{1,0,1,0}{Verde}
			\item \textcolor[cmyk]{1,1,0,0}{Azul}
			\item \textcolor[cmyk]{0,0,0,0}{Blanco}
			\item \textcolor[cmyk]{0,0,1,0}{Amarillo}
			\item \textcolor[cmyk]{0,1,0,0}{Magenta}
			\item \textcolor[cmyk]{1,0,1,0}{Rojo}
			\item \textcolor[cmyk]{1,0,0,0}{Cian}
		\end{itemize}
	\section{Modo matemático básico}
		Una ecuacion de segundo grado es :
			$ax^2+bx+c=0$ \\
		Una ecuación de segundo grado en otra linea
			$$ax^2+bx+c=0$$	
		Una sucesión de números se compone de :
		$ x_n = x_1 + x_2 +...... = x_n$
\end{document}
